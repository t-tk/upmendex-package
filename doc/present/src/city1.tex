%
% Example of upLaTeX, pxbabel, HaranoAji fonts and upmendex
%   in Latin, CJK (Chinese, Japanese, Korean) and Cyrillic
%
%
% Copyright (C) 2022 TANAKA Takuji
% You may freely use, modify and/or distribute this file.
%
\documentclass[a4paper]{ujarticle}

\usepackage[T2A,T1]{fontenc}
\usepackage{lmodern}
%\usepackage{tgtermes,tgheros}
\usepackage[japanese,korean,tchinese,schinese,english]{pxbabel}

\kcatcode`¥=15% U+00:¥ (Latin-1 Supplement)
\kcatcode`П=15% U+041F:П (Cyrillic)
\kcatcode`’=15% U+2019:’ (General Punctuation)
\kcatcode`€=15% U+20AC:€ (Currency Symbols)

\setlength{\textheight}{135mm}

\usepackage{makeidx}
\makeindex

\parindent1em
\pagestyle{empty}
\begin{document}
\thispagestyle{empty}
\section{Latin}
This is an English text.

%upmendex\index{upmendex}.
%Install of upmendex\index{upmendex!Install@Install of ---}.
%Usage of upmendex\index{upmendex!Usage@Usage of ---}.
%Usage of upmendex for beginners\index{upmendex!Usage@Usage of ---!for beginners@--- for beginners}.
%Usage of upmendex for experts\index{upmendex!Usage@Usage of ---!for experts@--- for experts}.

%Number: 3.14159265\index{3.14159265}, 2.71828182\index{2.71828182}.
%Symbol: \$\index{\$}, €\index{€}, ¥\index{¥}.
Number: 3.14159265\index{3.14159265}.
Symbol: €\index{€}.

São Paulo\index{São Paulo}.
Ciudad de México\index{Ciudad de México}.
%New York\index{New York}.
%Buenos Aires\index{Buenos Aires}.
İstanbul\index{İstanbul}.
%Rio de Janeiro\index{Rio de Janeiro}.
%London\index{London}.
%Manila\index{Manila}.
%Paris\index{Paris}.
%Berlin\index{Berlin}.

\section{CJK}
\subsection{Japanese (Kana, Kanji)}
これは日本語のテキストです。

%upmendex\index{upmendex}。
%upmendexのインストール\index{upmendex!のインストール@---のインストール}。
%upmendexの使い方\index{upmendex!のつかいかた@---の使い方}。
%upmendexの使い方入門編\index{upmendex!のつかいかた@---の使い方!にゅうもんへん@---入門編}。
%upmendexの使い方応用編\index{upmendex!のつかいかた@---の使い方!おうようへん@---応用編}。

東京\index{とうきょう@東京}。
%横浜\index{よこはま@横浜}。
大阪\index{おおさか@大阪}。
%名古屋\index{なごや@名古屋}。
札幌\index{さっぽろ@札幌}。
%福岡\index{ふくおか@福岡}。
%川崎\index{かわさき@川崎}。
%神戸\index{こうべ@神戸}。
%京都\index{きょうと@京都}。
さいたま\index{さいたま}。

\subsection{Korean (Hangul, Hanja)}
\begin{otherlanguage}{korean}
이것은 한국어 텍스트입니다.

서울\index{서울}.
%부산(釜山)\index{부산@부산(釜山)}.
%인천(仁川)\index{인천@인천(仁川)}.
대구(大邱)\index{대구@대구(大邱)}.
대전(大田)\index{대전@대전(大田)}.
%광주(光州)\index{광주@광주(光州)}.
%울산(蔚山)\index{울산@울산(蔚山)}.
평양(平壤)\index{평양@평양(平壤)}.
\end{otherlanguage}

\subsection{Chinese (Hanzi)}
\begin{otherlanguage}{tchinese}
這是中文文本。
\foreignlanguage{schinese}{这是中文文本。}

北京\index{北京}。
%上海\index{上海}。
%天津\index{天津}。
廈門\foreignlanguage{schinese}{(厦门)}\index{廈門@廈門\foreignlanguage{schinese}{(厦门)}}。
%深圳\index{深圳}。
%成都\index{成都}。
%杭州\index{杭州}。
%香港\index{香港}。
臺北\foreignlanguage{schinese}{(台北)}\index{臺北@臺北\foreignlanguage{schinese}{(台北)}}。
%高雄\index{高雄}。
%桃園\foreignlanguage{schinese}{(桃园)}\index{桃園@桃園\foreignlanguage{schinese}{(桃园)}}。
\end{otherlanguage}

\section{Cyrillic}
\fontencoding{T2A}\selectfont
Здесь русский текст.

Москва\index{Москва}.
%Санкт-Петербург\index{Санкт-Петербург}.
%Новосибирск\index{Новосибирск}.
Київ\index{Київ}.
%Одеса\index{Одеса}.
%Харків\index{Харків}.
%София\index{София}.
Београд\index{Београд}.
Бишкек\index{Бишкек}.
%Скопје\index{Скопје}.
%Мінск\index{Мінск}.
%Нұр-Сұлтан\index{Нұр-Сұлтан}.

\selectlanguage{english}
\fontencoding{T1}\selectfont

\printindex

\end{document}
